% !TeX spellcheck = fr_FR
\chapter*{Introduction}
\addcontentsline{toc}{chapter}{Introduction} % Adding toc entry

Depuis l'antiquité à nos jours, les cartes sont le reflet des premières formes de données récoltées par l'Homme.
La cartographie a aidé l'humanité à définir des chemins à travers le monde et à naviguer.
Cependant avec les fortes avancées technologiques des dernières décennies, les données récoltées ont augmenté massivement et notablement dans la cartographie avec les données LIDAR. Elles représentent un nuage de point dense qui sont actuellement une des informations collectées des plus massives.
Elles sont utilisées dans l'étude topographique de régions, dans les géosciences, dans la science environnementale ou bien encore elles viennent en aide au guidage automatique de véhicule terrestre. Un problème fréquent lié à ces données est le traitement réalisé avant leurs utilisations dans des applications, qui est souvent nécessaire afin de filtrer tout ce qui ne présente aucune utilité et pourrait même fausser les résultats.

Il reste cependant difficile de les traiter manuellement due à la quantité de données importante. Pour ce faire, des méthodes analytiques sont employées afin d'accélérer ce processus au travers de systèmes automatisés.
Un autre problème présent est leurs utilisations pour reconstruire des surfaces. Les méthodes automatisées n'étant pas à 100\% fiables, il est nécessaire de vérifier la cohérence des données ainsi que la qualité des maillages produits.
Un objectif supplémentaire est de réduire leurs empreintes volumiques sur les moyens de stockage.

Mon projet est une continuation d'un projet de semestre effectué sur la reconstruction de surface à partir de données lidar et se concentre plus spécifiquement sur le traitement des maillages générés ainsi que du traitement des données lidars avant l'étape de reconstruction de surface.
Le but principal étant de mettre à disposition des utilisateurs des outils de traitement efficace.
Il est aussi demandé de réaliser un client web permettant la visualisation des données lidars et des maillages.
Les données lidars sont misent à disposition par le \gls{sitg} et sont utilisés avec la triangulation de Delaunay pour créer des maillages reconstruisant les surfaces des régions.
Dans ce mémoire, nous allons en un premier temps voir et détailler les formats de fichier rencontrer dans l'application.
Ensuite nous détaillerons les méthodes algorithmiques mises en oeuvre pour réaliser des traitements sur ces fichiers.
Enfin nous allons voir comment cela a été implémenté dans l'application. Le lien du dépôt git est le suivant: \href{https://githepia.hesge.ch/jerome.chetelat/meshtools}{https://githepia.hesge.ch/jerome.chetelat/meshtools}