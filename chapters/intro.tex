% !TeX spellcheck = fr_FR
\chapter*{Introduction}
\addcontentsline{toc}{chapter}{Introduction} % Adding toc entry
\subsection*{Contexte}
Depuis l'antiquité à nos jours, les cartes sont le reflet des premières formes de données récoltées par l'Homme.
La cartographie a aidé l'humanité à définir des chemins à travers le monde et à naviguer.
Cependant avec les fortes avancées technologiques des dernières décennies,
les données récoltées ont augmenté massivement et notamment dans la cartographie avec les données LIDAR.
Ces dernières représentent un nuage de points dense qui sont actuellement une des informations collectées des plus massives.
Elles sont utilisées dans l'étude topographique de régions, dans les géosciences,
dans la science environnementale ou bien encore elles viennent en aide au guidage automatique de véhicule terrestre.
Un problème fréquent lié à ces données est le traitement réalisé avant leurs
utilisations dans des applications, qui est souvent nécessaire afin de nettoyer
tout ce qui ne présente aucune utilité et pourrait même fausser les résultats.
Il reste cependant difficile de les traiter manuellement due à la quantité de données importante.
Pour ce faire, des méthodes analytiques sont employées afin d'accélérer ce processus au travers de systèmes automatisés.
Un autre problème présent est leurs utilisations pour reconstruire des surfaces.
Les méthodes automatisées n'étant pas totalement fiables, il est nécessaire de
vérifier la cohérence des données ainsi que la qualité des reconstructions de
surface.

Ce travail de bachelor est une continuation d'un projet de semestre effectué sur
la reconstruction de surface à partir de données lidar. Les données lidar sont misent à disposition par le \gls{sitg} et sont utilisés
avec la triangulation de Delaunay pour créer des maillages reconstruisant les
surfaces des régions.

\subsection*{Buts}
Le but principal de ce travail est de créer une panoplie d'outils de traitement
de fichiers lidar ainsi que de fichier stl. Ils doivent permettre à un
utilisateur de reconstruire des surfaces grâce à une triangulation sur un nuage
de points, nettoyer des données lidar provenant d'un aéronef ainsi que
d'afficher un fichier stl dans un navigateur web. Un objectif supplémentaire est
de mettre en oeuvre ces outils grâce au langage de programmation Rust sur un
stack d'application entier (un seul langage pour le serveur et le client).

\subsection*{Méthodologie}
Dans un premier temps nous avons recherché des algorithmes de
traitement de données lidar. Nous avons par la suite implémenté ceux que nous
pensions judicieux d'être présent dans l'application.
Enfin nous avons testé ces algorithmes sur un jeu de données et fait des mesure
de performance.

Dans ce mémoire, nous allons en un premier détailler les formats de fichier 
rencontrés dans l'application.
Ensuite nous détaillerons les méthodes algorithmiques misent en oeuvre pour
réaliser des traitements sur lesdits fichiers.
Enfin nous allons voir comment ces dernières ont été implémenté dans
l'application et comparer leurs résultats.
Les sources des implémentation sont disponible à l'adresse: \href{https://githepia.esge.ch/jerome.chetelat/meshtools}{https://githepia.hesge.ch/jerome.chetelat/meshtools}
