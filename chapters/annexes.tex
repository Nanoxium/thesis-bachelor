% !TeX spellcheck = fr_FR
\addcontentsline{toc}{chapter}{Annexes} % Adding toc entry
%%% COMMENT THESES LINES IF YOU DO NOT USE DEDICATED TOC FOR ANNEXES
\stopcontents[default]
\resumecontents[annexes]
%%% /COMMENT THESES LINES IF YOU DO NOT USE DEDICATED TOC FOR ANNEXES
\chapter*{Annexes}

\begin{center}
\textit{Imprimer idéalement cette page sur une page de couleur.}
\textit{Chaque annexe doit commencer sur une nouvelle page et doit être numérotée : Annexe 1 puis Annexe 2, etc.}
\end{center}


\chapter*{Annexe 1: Tableau des champs de l'en-tête de fichier LAS}
\addcontentsline{toc}{chapter}{Annexe 1: Tableau des champs de l'en-tête de fichier LAS}
\begin{table}[htpb!]
\centering
\begin{tabular}{|l|l|c|c|}
\hline
\multicolumn{1}{|c|}{\textbf{Champ}} & \multicolumn{1}{c|}{\textbf{Type de donnée}} & \textbf{Taille (Octets)} & \textbf{Requis} \\ \hline
File Signature (“LASF”)              & char{[}4{]}              & 4     & * \\ \hline
File Source ID                       & unsigned short           & 2     & * \\ \hline
Global Encoding                      & unsigned short           & 2     &   \\ \hline
Project ID - GUID data 1             & unsigned long            & 4     &   \\ \hline
Project ID - GUID data 2             & unsigned short           & 2     &   \\ \hline
Project ID - GUID data 3             & unsigned short           & 2     &   \\ \hline
Project ID - GUID data 4             & unsigned char{[}8{]}     & 8     &   \\ \hline
Version Major                        & unsigned char            & 1     & * \\ \hline
Version Minor                        & unsigned char            & 1     & * \\ \hline
System Identifier                    & char{[}32{]}             & 32    & * \\ \hline
Generating Software                  & char{[}32{]}             & 32    & * \\ \hline
File Creation Day of Year            & unsigned short           & 2     &   \\ \hline
File Creation Year                   & unsigned short           & 2     &   \\ \hline
Header Size                          & unsigned short           & 2     & * \\ \hline
Offset to point data                 & unsigned long            & 4     & * \\ \hline
Number of Variable Length Records    & unsigned long            & 4     & * \\ \hline
Point Data Format ID (0-99 for spec) & unsigned char            & 1     & * \\ \hline
Point Data Record Length             & unsigned short           & 2     & * \\ \hline
Number of point records              & unsigned long            & 4     & * \\ \hline
Number of points by return           & unsigned long{[}5{]}     & 20    & * \\ \hline
X scale factor                       & double                   & 8     & * \\ \hline
Y scale factor                       & double                   & 8     & * \\ \hline
Z scale factor                       & double                   & 8     & * \\ \hline
X offset                             & double                   & 8     & * \\ \hline
Y offset                             & double                   & 8     & * \\ \hline
Z offset                             & double                   & 8     & * \\ \hline
Max X                                & double                   & 8     & * \\ \hline
Min X                                & double                   & 8     & * \\ \hline
Max Y                                & double                   & 8     & * \\ \hline
Min Y                                & double                   & 8     & * \\ \hline
Max Z                                & double                   & 8     & * \\ \hline
Min Z                                & double                   & 8     & * \\ \hline
\end{tabular}
\caption[Champs de l'en-tête d'un fichier LAS]{Champs de l'en-tête d'un fichier LAS}
\label{tab:las_header}
\end{table}

\chapter*{Annexe 2}
\addcontentsline{toc}{chapter}{Annexe 2}


\chapter*{Annexe 3}
\addcontentsline{toc}{chapter}{Annexe 3}


%%% COMMENT THESES LINES IF YOU DO NOT USE DEDICATED TOC FOR ANNEXES
\stopcontents[annexes]
\resumecontents[default]
%%% /COMMENT THESES LINES IF YOU DO NOT USE DEDICATED TOC FOR ANNEXES