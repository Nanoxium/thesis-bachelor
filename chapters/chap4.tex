% !TeX spellcheck = fr_FR

\chapter{Chapitre 4 : Résultats}
Dans ce chapitre, nous allons voir les résultats obtenus par les
différentes parties de l'application. On va aussi comparer les différentes
métriques collectés des binaires.

Les tests suivants ont été réaliser principalement sur la configuration d'ordinateur
se trouvant dans la figure \ref{fig:computer_configuration} en annexe 4.

\section{Triangulation}

La triangulation de Delaunay provient en grande partie du projet de semestre.
Seule la partie fusion de maillage a été implémenté dans ce travail.

Pour la triangulation, on utilise deux fichiers.
\begin{itemize}
	\item Le fichier 2501500_1112000.las du jeu de données des \gls{sitg} contenant 9'000'987 points
	\item Un fichier contenant que les 100'000 premiers points du fichier précédent.
\end{itemize}

\subsection{Algorithme de Bowyer-Watson}


La commande prend en moyenne 6.55 

\subsection{Fusion de triangulation}

\section{Nettoyage}

\section{Stack Web}
\subsection{Serveur Web}

Le serveur actuel permet les téléchargements de fichiers \gls{stl} présent dans
un dossier spécifique. 

\subsection{Client Web}
Les tests suivants ont été effectué sur les navigateur Firefox version 81 et
Chromium version.
Le client web permet actuellement de faire un rendu d'un maillage dans un
navigateur web. Un fichier \gls{stl} est téléchargé à l'aide de l'api
\textit{fetch} du navigateur au chargement de la page web.

\textbf{Amélioration possible} \\
Le chargement des vertex du maillage se faisant dans un seul file d'exécution,
la page lors du chargement reste figée et peut ne plus répondre (dépend de la
puissance de l'hôte) .
Il serait possible dans un futur proche de faire ces opérations dans des WebWorkers. Au
moment de l'écriture de ce mémoire, les possibilités d'utiliser des WebWorkers
en Rust est encore expérimental.
