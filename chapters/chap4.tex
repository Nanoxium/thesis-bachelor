% !TeX spellcheck = fr_FR

\chapter{Chapitre 4 : Résultats}
Dans ce chapitre, nous allons détailler les résultats obtenus par les
différentes partie de l'application.
\section{Triangulation}

La triangulation implémentée nous provient en partie du projet de semestre.

\section{Nettoyage}

\section{Stack Web}
\subsection{Serveur Web}

Le serveur actuel permet les téléchargements de fichiers \gls{stl} présent dans
un dossier spécifique. 

\subsection{Client Web}
Les tests suivants ont été effectué sur le navigateur Firefox version 81.
Le client web permet actuellement de faire un rendu d'un maillage dans un
navigateur web. Un fichier \gls{stl} est téléchargé à l'aide de l'api
\textit{fetch} du navigateur au chargement de la page web.

\textbf{Amélioration possible} \\
Le chargement des vertex du maillage se faisant dans un seul file d'exécution,
la page lors du chargement reste figée et peut ne plus répondre. Il serait
possible dans un futur proche de faire ces opérations dans des WebWorkers. Au
moment de l'écriture de ce mémoire, les possibilités d'utiliser des WebWorkers
en Rust est encore expérimental.
