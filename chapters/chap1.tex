% !TeX spellcheck = fr_FR
\chapter{Chapitre 1 : Données et formats}

\begin{center}
	\textit{Chaque chapitre doit commencer sur une nouvelle page.}\\*[35pt]
\end{center}

Dans ce chapitre nous allons aborder les différents format de données qui sont
cités et utilisé dans la suite de ce document.

\section{Données LiDAR}

% \begin{figure}[tbph!]
% 	\centering
% 	\includegraphics[width=0.7\linewidth]{chart}
% 	\caption[Diagramme machin]{Diagramme machin. Source : tiré de Tartempion 2010, p. 42 / tiré de ce-site.ch, ref. URL01 / réalisé par Nom Prénom.}
% 	\label{fig:chart1}
% \end{figure}
% 
% 
% \begin{table}[tbph!]
% 	\centering{
% 		\begin{tabular}{ |l|c|c|c| }
% 			\hline
% 			& \textbf{Condition 1} & \textbf{Condition 2} & \textbf{Condition 3} \\
% 			\hline
% 			\textbf{Test 1} & X & O & X \\
% 			\hline
% 			\textbf{Test 2} & O & X & X \\
% 			\hline
% 			\textbf{Test 3} & O & X & O \\
% 			\hline 
% 		\end{tabular}
% 		\caption[Lot de données n°1]{Lot de données n°1. Source: tiré de Tartempion 2010, p. 42 / tiré de ce-site.ch, ref. URL02 / réalisé par Nom Prénom.}
% 		\label{tab:tableau1}
% 	}
% \end{table}
% 
% 
% \begin{figure}[tbph!]
% 	\centering
% 	\includegraphics[width=0.7\linewidth]{diagram}
% 	\caption[Schéma bidule.]{Schéma bidule. Source : tiré de Tartempion 2010, p. 42 / tiré de ce-site.ch, ref. URL03 / réalisé par Nom Prénom.}
% 	\label{fig:diagram}
% \end{figure}
% 
% 
% \section{Titre de niveau 2}
% 
% 
% \begin{figure}[tbph!]
% 	\centering
% 	\includegraphics[width=0.7\linewidth]{ordi}
% 	\caption[Alice, Micro-ordinateur MATRA.]{Alice, Micro-ordinateur MATRA. Source : tiré de Tartempion 2010, p. 42 / tiré de ce-site.ch, ref. URL03 / réalisé par Nom Prénom.}
% 	\label{fig:image}
% \end{figure}
% 
% 
% \subsection{Titre de niveau 3}
% 
% 
% \begin{table}[tbph!]
% 	\centering{
% 		\begin{tabular}{ |l|c|c|c| }
% 			\hline
% 			& \textbf{Condition 1} & \textbf{Condition 2} & \textbf{Condition 3} \\
% 			\hline
% 			\textbf{Test 1} & X & O & X \\
% 			\hline
% 			\textbf{Test 2} & O & X & X \\
% 			\hline
% 			\textbf{Test 3} & O & X & O \\
% 			\hline 
% 		\end{tabular}
% 		\caption[Lot de données n°2.]{Lot de données n°2. Source: tiré de Tartempion 2010, p. 42 / tiré de ce-site.ch, ref. URL05 / réalisé par Nom Prénom.}
% 		\label{tab:tableau2}
% 	}
% \end{table}
