% !TeX spellcheck = fr_FR
\chapter*{Conclusion}
\addcontentsline{toc}{chapter}{Conclusion} % Adding toc entry

\textit{TODO}
 Ce travail avait pour objectif de créer et de mettre à disposition des outils pour manipuler les données \gls{lidar} ainsi qu'une interface web pour afficher certaines données.
Les outils devaient pouvoir nettoyer des données \gls{lidar} et les trianguler de manière efficace, de filtrer et fusionner des maillages.
Certaines fonctionnalités ont été implémentées dans le langage Rust permettant de compiler pour des plateformes inhabituelles pour un langage système tel que le web.
Les outils présents ne sont pas tous 100\% fonctionnels.
Certains outils comme le client web sont encore inachevés et nécessite des corrections pour fonctionner correctement.
La réalisation de ce travail m'a permis de mettre en oeuvre l'enseignement reçu lors de ma formation ainsi que d'enrichir mes connaissances dans le domaine de la topographie numérique. J'ai pu aussi utiliser le langage Rust que je compte désormais utiliser de manière régulière pour mes projets personnels, car l'écosystème, malgré son manque de maturité, rend le développement de logiciels facile et amusant. Ce travail était aussi un challenge, non seulement à cause du covid19, mais aussi de par l'ampleur du projet qui dépasse tout ce que j'ai pu rencontrer auparavant.

Le travail étant inachevé, un futur travail pourrait reprendre le projet et terminer ce qui n'est pas terminé, mais aussi rajouter de nouvelles fonctionnalités comme la gestion de nouveau format de données de maillages ou encore l'optimisation de certains algorithmes implémentés pour gagner en performance.