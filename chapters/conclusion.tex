% !TeX spellcheck = fr_FR
\chapter*{Conclusion}
\addcontentsline{toc}{chapter}{Conclusion} % Adding toc entry

Ce travail avait pour objectif de créer et de mettre à disposition des outils pour manipuler les données \gls{lidar} ainsi qu'une interface web pour les afficher.
Les outils devais pouvoir filtrer des données LIDAR et les trianguler de manière efficace, de filtrer et fusionné des maillages et pour cela, une travail de recherche sur les formats de données a été réalisée pour tiré meilleurs partie des informations à disposition.
Les fonctionnalités ont été implémentées dans le langage Rust et grâce à cela, les différents outils bénéficies d'une garantie de ne pas avoir de fuite mémoire.
L'interface web a été implémentée aussi en Rust et compilé en \gls{wasm} afin de bénéficier de performances supplémentaires pour l'affichage des jeux de données. 
La réalisation de ce travail, m'a permis, en plus de mettre en oeuvre l'enseignement reçu lors de ma formation mais encore d'avoir pu enrichir mes connaissances dans le domaine de la topographie et du langage de programmation Rust que je compte désormais approfondir encore plus. Un challenge particulier lors de ce travail était le confinement du au covid19. Mon environnement de travail n'était pas propice à la productivité due aux bruits des personnes vivant sous le même toit me gênant dans ma concentration.

Le travail n'étant pas complètement terminé, notamment aux niveaux du client web, un futur travail pourrait reprendre le projet et continuer de l'améliorer avec des fonctionnalités stables. Des outils supplémentaires peuvent être ajouter pour compléter la panoplie d'utilisation de l'application.