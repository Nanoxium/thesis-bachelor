% !TeX spellcheck = fr_FR
\chapter*{Conclusion}
\addcontentsline{toc}{chapter}{Conclusion} % Adding toc entry

\textit{TODO}
% Ce travail avait pour objectif de créer et de mettre à disposition des outils pour manipuler les données \gls{lidar} ainsi qu'une interface web pour les afficher.
% Les outils devaient pouvoir filtrer des données LIDAR et les trianguler de manière efficace, de filtrer et fusionné des maillages et pour cela, un travail de recherche sur les formats de données a été réalisé pour tiré meilleure partie des informations à disposition.
% Les fonctionnalités ont été implémentées dans le langage Rust et grâce à cela, les différents outils bénéficient d'une garantie de ne pas avoir de fuites mémoires.
% L'interface web a été implémentée aussi en Rust et compilée en \gls{wasm} afin de bénéficier de performances supplémentaires pour l'affichage des jeux de données. 
% La réalisation de ce travail, m'a permis, en plus de mettre en oeuvre l'enseignement reçu lors de ma formation, mais encore d'avoir pu enrichir mes connaissances dans le domaine de la topographie et du langage de programmation Rust que je compte désormais approfondir encore plus. Un challenge particulier lors de ce travail était le confinement dû au covid19. Mon environnement de travail n'était pas propice à la productivité due aux bruits des personnes vivant sous le même toit me gênant dans ma concentration. Ce fut un gros challenge de devoir travailler uniquement à la maison sans pouvoir avoir du calme.
% 
% Le travail n'étant pas complètement terminé, notamment au niveau du client web, un futur travail pourrait reprendre le projet et continuer de l'améliorer avec des fonctionnalités stables. Des outils supplémentaires peuvent être ajoutés pour compléter la panoplie de l'application notamment au travers de services web.
