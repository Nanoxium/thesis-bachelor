% !TeX spellcheck = fr_FR
\chapter*{Conclusion}
\addcontentsline{toc}{chapter}{Conclusion} % Adding toc entry

\textit{TODO}
Ce travail avait pour objectif de créer et de mettre à disposition des outils pour manipuler les données \gls{lidar} ainsi qu'une interface web pour afficher certaines données.
Les outils devaient pouvoir nettoyer des données \gls{lidar} et les trianguler de manière efficace, de filtrer et fusionner des maillages.
Pour cela, on a du en premier explorer les formats de fichier \gls{lidar} et
\gls{stl}.
Ensuite nous avons explorer les algorithmes qui allait être implémentées dans
l'application comme l'algorithme de Bowyer-Watson qui permet de faire une
triangulation de Delaunay et l'algorithme de Lee \& Schachter pour la fusion de
triangulation. Nous avons aussi vu les différentes méthodes de nettoyage de
données pour réduire la taille des fichiers utilisés et éviter de fausser les
résultat dans leur utilisation ultérieur. Par la suite nous avons implémenté ces
méthodes dans des outils en ligne de commande qui ont été implémentées dans le langage Rust.
Ce dernier permet de compiler pour des plateformes inhabituelles pour un langage système tel que le web et grâce à cela,
nous avons réaliser un serveur de fichier et un client web pour afficher des
maillages.
Nous avons ensuite testé nos outils et comparé les résultats obtenus et
vu que les méthodes de nettoyage donnait des résultats numériques similaire mais
visuels différents. L'algorithme de Bowyer-Watson donnait un résultat pour
un nombre de points limité mais lorsqu'on passait à un jeu de donnée complet, aucun
résultat n'a été trouvé au vu du temps de calculs trop important. 
On avait vu que le serveur de fichier servait des fichiers au client web qui les
affiche à l'aide de WebGL.
Certains outils comme le client web sont encore inachevés et nécessitent des corrections pour fonctionner correctement.
La réalisation de ce travail m'a permis de mettre en oeuvre l'enseignement reçu lors de ma formation ainsi que d'enrichir mes connaissances dans le domaine de la topographie numérique.
J'ai pu aussi utiliser le langage Rust que je compte désormais utiliser de manière régulière pour mes projets personnels, car l'écosystème, malgré son manque de maturité, rend le développement de logiciels facile et amusant.
Ce travail était aussi un challenge, non seulement à cause du covid19, mais aussi de par l'ampleur du projet qui dépasse tout ce que j'ai pu rencontrer auparavant.

Le travail étant inachevé, un futur travail pourrait reprendre le projet et
terminer ce qui n'est pas encore complet, mais aussi rajouter de nouvelles
fonctionnalités comme la gestion de nouveaux formats de données de maillages ou encore l'optimisation de certains algorithmes implémentés afin de pouvoir véritablement gagner en performance.
